\documentclass{article}
\title{Problem set, MT5017}
\author{Ville Sebastian Olsson}
\usepackage[a4paper,margin=2em]{geometry}
\usepackage{amsmath}
\usepackage{amsfonts}
\usepackage{amssymb}
\usepackage{amsthm}
\usepackage{dsfont}
\usepackage{mathtools}
\usepackage[parfill]{parskip}
\usepackage{hyperref} % Clickable ToC
\usepackage{tikz}
\usepackage[breakable]{tcolorbox}
\usepackage{graphicx}
\usepackage[utf8]{inputenc}
\usepackage{booktabs}
\usepackage{csvsimple}
\usepackage{float}
\usepackage{longtable}
\usepackage{csquotes}
\usepackage{listings}
\graphicspath{ {./images/} }
\hypersetup{colorlinks}

\usepackage{sebelino-mathlib} % Custom sty file
\setcounter{secnumdepth}{0} % Disable numbering

\lstset{
  language=Python,
  basicstyle=\ttfamily\footnotesize,
  keywordstyle=\color{blue},
  stringstyle=\color{red},
  commentstyle=\color{green!50!black},
  backgroundcolor=\color{yellow!10},
  frame=single,
  numbers=left,
  numberstyle=\tiny\color{gray},
  tabsize=4,
  showstringspaces=false,
  captionpos=b
}

\begin{document}
\maketitle
\tableofcontents

\section{W1}

\subsection{W1:1}

Consider an \(n\times n\) grid.
Then \(T_n\) is the set of paths from \((0,0)\) to \((n,n)\),
subject to the condition that the initial edge is pointing rightwards.

Then \(T_n\) is the set of paths from \((1,0)\) to \((n,n)\).

Then \(T_n\) is the set of vectors \(\mathbf{x} = (x_1,\ldots,x_{2n-1})\),
where \(x_i\in \{-1,1\}\),
such that \(1+\sum_{i=1}^{2n-1} x_i = 0\).
\[T_n = \{ \mathbf{x}\in \{-1,1\}^{2n-1}: \sum_{i=1}^{2n-1} x_i = 0\}\]

Then there are \(n-1\) \(x_i\) that are equal to 1:
\[|\{i \in \{1,\ldots,2n-1\}: x_i = 1\}| = n-1\]

We want to find \(|T_n|\). This is equal to the number of way to choose \(n-1\) balls
from an urn containing \(2n-1\) balls:
\[|T_n| = \binom{2n-1}{n-1}\]

\subsection{W1:2}

Idea: Generate a large set of paths in \(T_n\), then
count the number of paths that satisfy the condition of being a member of \(S_n\).

\begin{minipage}{\linewidth}
\begin{lstlisting}
   Monte Carlo Estimate
2                2.0163
3                5.0290
4               13.9020
\end{lstlisting}
\end{minipage}

\[c_2^{10000} \approx 2\]
\[c_3^{10000} \approx 5\]
\[c_4^{10000} \approx 14\]


\end{document}
