\documentclass{article}
\title{Problem set, MT5017}
\author{Ville Sebastian Olsson}
\usepackage[a4paper,margin=2em]{geometry}
\usepackage{amsmath}
\usepackage{amsfonts}
\usepackage{amssymb}
\usepackage{amsthm}
\usepackage{dsfont}
\usepackage{mathtools}
\usepackage[parfill]{parskip}
\usepackage{hyperref} % Clickable ToC
\usepackage{tikz}
\usepackage[breakable]{tcolorbox}
\usepackage{graphicx}
\usepackage[utf8]{inputenc}
\usepackage{booktabs}
\usepackage{csvsimple}
\usepackage{float}
\usepackage{longtable}
\usepackage{csquotes}
\usepackage{listings}
\graphicspath{ {./images/} }
\hypersetup{colorlinks}

\usepackage{sebelino-mathlib} % Custom sty file
\setcounter{secnumdepth}{0} % Disable numbering

\lstset{
  language=Python,
  basicstyle=\ttfamily\footnotesize,
  keywordstyle=\color{blue},
  stringstyle=\color{red},
  commentstyle=\color{green!50!black},
  backgroundcolor=\color{yellow!10},
  frame=single,
  numbers=left,
  numberstyle=\tiny\color{gray},
  tabsize=4,
  showstringspaces=false,
  captionpos=b
}

\begin{document}
\maketitle
\tableofcontents

\section{W1}

\subsection{W1:1a}

Consider an \(n\times n\) grid.
Then \(T_n\) is the set of paths from \((0,0)\) to \((n,n)\),
subject to the condition that the initial edge is pointing rightwards.

Then \(T_n\) is the set of paths from \((1,0)\) to \((n,n)\).

Then \(T_n\) is the set of vectors \(\mathbf{x} = (x_1,\ldots,x_{2n-1})\),
where \(x_i\in \{-1,1\}\),
such that \(1+\sum_{i=1}^{2n-1} x_i = 0\).
\[T_n = \{ \mathbf{x}\in \{-1,1\}^{2n-1}: \sum_{i=1}^{2n-1} x_i = 0\}\]

Then there are \(n-1\) \(x_i\) that are equal to 1:
\[|\{i \in \{1,\ldots,2n-1\}: x_i = 1\}| = n-1\]

We want to find \(|T_n|\). This is equal to the number of way to choose \(n-1\) balls
from an urn containing \(2n-1\) balls:
\[|T_n| = \binom{2n-1}{n-1}\]

\subsection{W1:1b}

Idea: Generate a large set of paths in \(T_n\), then
count the number of paths that satisfy the condition of being a member of \(S_n\).

\begin{minipage}{\linewidth}
\begin{lstlisting}
   Monte Carlo Estimate
2                1.9923
3                4.9730
4               13.9825
   |T_n|
2      3
3     10
4     35
\end{lstlisting}
\end{minipage}

\[c_2^{10000} \approx 2\]
\[c_3^{10000} \approx 5\]
\[c_4^{10000} \approx 14\]

\subsection{W1:1c}

\begin{align*}
   & c_n \\
   =& |S_n| \\
   =& |T_n|\cdot P(X_1\text{ valid path}) \\
   =& |T_n|\cdot P(X_1\in S_n) \\
   \approx& |T_n|\cdot \hat{P}_N(X_1\in S_n) \\
   =& |T_n|\cdot \frac{1}{N}\sum_{i=1}^N\mathds{1}_{S_n}(X_i) \\
   =& |T_n|\cdot \frac{1}{N}\sum_{i=1}^N Y_i & (Y_i=\mathds{1}_{S_n}(X_i)\sim \text{Be}\left(\frac{|S_n|}{|T_n|}\right)) \\
   =& |T_n|\cdot \bar{Y}_N \\
   =& \dot{c}_n^N \\
\end{align*}

Objective function:
\[\phi(X_1) = \mathds{1}_{S_n}(X_1)\]
Expectation:
\[E[\phi(X_1)] = E[\mathds{1}_{S_n}(X_1)] = P(X_1 \in S_n) = \frac{|S_n|}{|T_n|}\]

CLT:
\begin{align*}
   & \sqrt{N}\frac{\bar{Y}_N-E[Y_1]}{\sqrt{\text{Var}(Y_1)}} \xrightarrow[N\to\infty]{d} \mathcal{N}(0,1) \\
   \Rightarrow& \frac{\bar{Y}_N-\frac{|S_n|}{|T_n|}}{\sqrt{\frac{|S_n|}{|T_n|}(1-\frac{|S_n|}{|T_n|})}} \xrightarrow[N\to\infty]{d} \mathcal{N}(0,\frac{1}{N}) \\
   \Rightarrow& \bar{Y}_N-\frac{|S_n|}{|T_n|} \xrightarrow[N\to\infty]{d} \mathcal{N}(0,\frac{1}{N}\frac{|S_n|}{|T_n|}(1-\frac{|S_n|}{|T_n|})) \\
   \Rightarrow& \bar{Y}_N \xrightarrow[N\to\infty]{d} \mathcal{N}\left(\frac{|S_n|}{|T_n|},\frac{|S_n|}{N|T_n|}(1-\frac{|S_n|}{|T_n|})\right) \\
   \Rightarrow& |T_n|\cdot\bar{Y}_N \xrightarrow[N\to\infty]{d} \mathcal{N}\left(|S_n|,\frac{|S_n|}{N}(|T_n|-|S_n|)\right) \\
   \Rightarrow& \dot{c}_n^N \xrightarrow[N\to\infty]{d} \mathcal{N}\left(|S_n|,\frac{|S_n|}{N}(|T_n|-|S_n|)\right) \\
\end{align*}

\(|T_n|\) is known.
\(|S_n|\) is unknown, but we can estimate it with \(c_n^N\).
\[\dot{c}_n^N \xrightarrow[N\to\infty]{d} \mathcal{N}\left(c_n^N,\frac{c_n^N}{N}(|T_n|-c_n^N)\right)\]

95 \% Wald CI:
\[c_n^N \pm z_{0.975}\sqrt{\frac{c_n^N}{N}(|T_n|-c_n^N)}\]
\[c_n^N \pm 1.96\sqrt{\frac{c_n^N}{N}(|T_n|-c_n^N)}\]

For \(n=2\):
\[c_2^{10000} \pm 1.96\sqrt{\frac{c_2^{10000}}{10000}(|T_2|-c_n^{10000})}\]
\[1.9923 \pm 1.96\sqrt{\frac{1.9923}{10000}(3-1.9923)}\]
\[[1.9645,2.0201]\]

For \(n=3\):
\[[4.8750,5.0710]\]

For \(n=4\):
\[[13.6465,14.3185]\]

\subsection{W1:1d}

Catalan numbers:

\[c_2^{10000} = 1.9923 \approx 2\]
\[c_3^{10000} = 4.9730 \approx 5\]
\[c_4^{10000} = 13.9825 \approx 14\]

























\end{document}
