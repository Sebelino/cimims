\documentclass{article}
\title{Problem set, MT5017}
\author{Ville Sebastian Olsson}
\usepackage[b5paper,margin=1em]{geometry}
\usepackage{amsmath}
\usepackage{amsfonts}
\usepackage{amssymb}
\usepackage{amsthm}
\usepackage{dsfont}
\usepackage{mathtools}
\usepackage[parfill]{parskip}
\usepackage{hyperref} % Clickable ToC
\usepackage{tikz}
\usepackage[breakable]{tcolorbox}
\usepackage{graphicx}
\usepackage[utf8]{inputenc}
\usepackage{booktabs}
\usepackage{csvsimple}
\usepackage{float}
\usepackage{longtable}
\usepackage{csquotes}
\usepackage{listings}
\graphicspath{ {./images/} }
\hypersetup{colorlinks}

\usepackage{sebelino-mathlib} % Custom sty file
\setcounter{secnumdepth}{0} % Disable numbering

\lstset{
  language=Python,
  basicstyle=\ttfamily\footnotesize,
  keywordstyle=\color{blue},
  stringstyle=\color{red},
  commentstyle=\color{green!50!black},
  backgroundcolor=\color{yellow!10},
  frame=single,
  numbers=left,
  numberstyle=\tiny\color{gray},
  tabsize=4,
  showstringspaces=false,
  captionpos=b
}

\begin{document}
\maketitle
\tableofcontents

\section{W1}

\subsection{W1:1a}

Consider an \(n\times n\) grid.
Then \(T_n\) is the set of paths from \((0,0)\) to \((n,n)\),
subject to the condition that the initial edge is pointing rightwards.

Then \(T_n\) is the set of paths from \((1,0)\) to \((n,n)\).

Then \(T_n\) is the set of vectors \(\mathbf{x} = (x_1,\ldots,x_{2n-1})\),
where \(x_i\in \{-1,1\}\),
such that \(1+\sum_{i=1}^{2n-1} x_i = 0\).
\[T_n = \{ \mathbf{x}\in \{-1,1\}^{2n-1}: \sum_{i=1}^{2n-1} x_i = 0\}\]

Then there are \(n-1\) \(x_i\) that are equal to 1:
\[|\{i \in \{1,\ldots,2n-1\}: x_i = 1\}| = n-1\]

We want to find \(|T_n|\). This is equal to the number of way to choose \(n-1\) balls
from an urn containing \(2n-1\) balls:
\[|T_n| = \binom{2n-1}{n-1}\]

The reason for having an initial right-edge is for efficiency. It is guaranteed
that \(S_n\) will start with a right-edge (and edge pointing upwards).

\subsection{W1:1b}

Idea: Generate a large set of paths in \(T_n\), then
count the number of paths that satisfy the condition of being a member of \(S_n\).

\begin{minipage}{\linewidth}
\begin{lstlisting}
   Monte Carlo Estimate
2                1.9923
3                4.9730
4               13.9825
   |T_n|
2      3
3     10
4     35
\end{lstlisting}
\end{minipage}

\[c_2^{10000} \approx 2\]
\[c_3^{10000} \approx 5\]
\[c_4^{10000} \approx 14\]

\subsection{W1:1c}

\begin{align*}
   & c_n \\
   =& |S_n| \\
   =& |T_n|\cdot P(X_1\text{ valid path}) \\
   =& |T_n|\cdot P(X_1\in S_n) \\
   \approx& |T_n|\cdot \hat{P}_N(X_1\in S_n) \\
   =& |T_n|\cdot \frac{1}{N}\sum_{i=1}^N\mathds{1}_{S_n}(X_i) \\
   =& |T_n|\cdot \frac{1}{N}\sum_{i=1}^N Y_i & (Y_i=\mathds{1}_{S_n}(X_i)\sim \text{Be}\left(\frac{|S_n|}{|T_n|}\right)) \\
   =& |T_n|\cdot \bar{Y}_N \\
   =& \dot{c}_n^N \\
\end{align*}

Objective function:
\[\phi(X_1) = \mathds{1}_{S_n}(X_1)\]
Expectation:
\[E[\phi(X_1)] = E[\mathds{1}_{S_n}(X_1)] = P(X_1 \in S_n) = \frac{|S_n|}{|T_n|}\]

CLT:
\begin{align*}
   & \sqrt{N}\frac{\bar{Y}_N-E[Y_1]}{\sqrt{\text{Var}(Y_1)}} \xrightarrow[N\to\infty]{d} \mathcal{N}(0,1) \\
   \Rightarrow& \frac{\bar{Y}_N-\frac{|S_n|}{|T_n|}}{\sqrt{\frac{|S_n|}{|T_n|}(1-\frac{|S_n|}{|T_n|})}} \xrightarrow[N\to\infty]{d} \mathcal{N}(0,\frac{1}{N}) \\
   \Rightarrow& \bar{Y}_N-\frac{|S_n|}{|T_n|} \xrightarrow[N\to\infty]{d} \mathcal{N}(0,\frac{1}{N}\frac{|S_n|}{|T_n|}(1-\frac{|S_n|}{|T_n|})) \\
   \Rightarrow& \bar{Y}_N \xrightarrow[N\to\infty]{d} \mathcal{N}\left(\frac{|S_n|}{|T_n|},\frac{|S_n|}{N|T_n|}(1-\frac{|S_n|}{|T_n|})\right) \\
   \Rightarrow& |T_n|\cdot\bar{Y}_N \xrightarrow[N\to\infty]{d} \mathcal{N}\left(|S_n|,\frac{|S_n|}{N}(|T_n|-|S_n|)\right) \\
   \Rightarrow& \dot{c}_n^N \xrightarrow[N\to\infty]{d} \mathcal{N}\left(|S_n|,\frac{|S_n|}{N}(|T_n|-|S_n|)\right) \\
\end{align*}

\(|T_n|\) is known.
\(|S_n|\) is unknown, but we can estimate it with \(c_n^N\).
\[\dot{c}_n^N \xrightarrow[N\to\infty]{d} \mathcal{N}\left(c_n^N,\frac{c_n^N}{N}(|T_n|-c_n^N)\right)\]

95 \% Wald CI:
\[c_n^N \pm z_{0.975}\sqrt{\frac{c_n^N}{N}(|T_n|-c_n^N)}\]
\[c_n^N \pm 1.96\sqrt{\frac{c_n^N}{N}(|T_n|-c_n^N)}\]

For \(n=2\):
\[c_2^{10000} \pm 1.96\sqrt{\frac{c_2^{10000}}{10000}(|T_2|-c_n^{10000})}\]
\[1.9923 \pm 1.96\sqrt{\frac{1.9923}{10000}(3-1.9923)}\]
\[[1.9645,2.0201]\]

For \(n=3\):
\[[4.8750,5.0710]\]

For \(n=4\):
\[[13.6465,14.3185]\]

\subsection{W1:1d}

Catalan numbers:

\[c_2^{10000} = 1.9923 \approx 2\]
\[c_3^{10000} = 4.9730 \approx 5\]
\[c_4^{10000} = 13.9825 \approx 14\]

\subsection{W1:2a}

\[w_\ell = P(X\in\text{X}_\ell) = \int_{\text{X}_\ell} f(x)dx\]
\[f_\ell(x) = f_{X\in\text{X}_\ell}(x) = \frac{1}{w_\ell} f(x)\mathds{1}_{\text{X}_\ell}(x)\]
\begin{align*}
   & \tau \\
   =& \int_{\text{X}} \phi(x)f(x)dx \\
   =& \sum_{\ell=1}^n\int_{\text{X}_\ell} \phi(x)f(x)dx \\
   =& \sum_{\ell=1}^n w_\ell\int_{\text{X}_\ell} \phi(x)\frac{1}{w_\ell}f(x)dx \\
   =& \sum_{\ell=1}^n w_\ell\int_{\text{X}} \phi(x)\frac{1}{w_\ell}f(x)\mathds{1}_{\text{X}_\ell}(x)dx \\
   =& \sum_{\ell=1}^n w_\ell\int_{\text{X}} \phi(x)f_\ell(x)dx \\
   =& \sum_{\ell=1}^n w_\ell \tau^{(\ell)} \\
\end{align*}

\[\tau^{(\ell)} = E[\phi(X)\,|\,X\in\text{X}_\ell] = \int_{\text{X}}\phi(x)f_\ell(x)dx\]

\[\tau_{N_\ell}^{(\ell)} = \frac{1}{N_\ell} \sum_{i=1}^{N_\ell} \phi(X|X\in \text{X}_\ell)\]

\begin{align*}
   & \sigma_\ell^2(\phi) \\
   =& \text{Var}(\phi(X|X\in\text{X}_\ell)) \\
   =& E[(\phi(X|X\in\text{X}_\ell)-E[\phi(X|X\in\text{X}_\ell)])^2)] \\
   =& E[(\phi(X|X\in\text{X}_\ell)-E[\phi(X)|X\in\text{X}_\ell])^2)] \\
   =& E[(\phi(X|X\in\text{X}_\ell)-\tau^{(\ell)})^2)] \\
   =& E[(\phi(X)-\tau^{(\ell)})^2|X\in\text{X}_\ell] \\
   =& \int_\text{X}(\phi(x)-\tau^{(\ell)})^2 f_{X|X\in\text{X}_\ell}(x) dx \\
   =& \int_\text{X}(\phi(x)-\tau^{(\ell)})^2 f_\ell(x) dx \\
\end{align*}


\begin{align*}
   & \text{Var}(\tilde{\tau}_N) \\
   =& \text{Var}\left(\sum_{\ell=1}^n w_\ell\tau_{N_\ell}^{(\ell)}\right) \\
   =& \sum_{\ell=1}^n \text{Var}\left(w_\ell\tau_{N_\ell}^{(\ell)}\right) \\
   =& \sum_{\ell=1}^n w_\ell^2\text{Var}\left(\tau_{N_\ell}^{(\ell)}\right) \\
   =& \sum_{\ell=1}^n w_\ell^2\text{Var}\left( \frac{1}{N_\ell} \sum_{i=1}^{N_\ell} \phi(X_i|X_i\in\text{X}_\ell) \right) \\
   =& \sum_{\ell=1}^n \frac{w_\ell^2}{N_\ell^2}\text{Var}\left( \sum_{i=1}^{N_\ell} \phi(X_i|X_i\in\text{X}_\ell) \right) \\
   =& \sum_{\ell=1}^n \frac{w_\ell^2}{N_\ell^2}\sum_{i=1}^{N_\ell} \text{Var}\left( \phi(X_i|X_i\in\text{X}_\ell) \right) \\
   =& \sum_{\ell=1}^n \frac{w_\ell^2}{N_\ell^2}\sum_{i=1}^{N_\ell} \sigma_\ell^2(\phi) \\
   =& \sum_{\ell=1}^n \frac{w_\ell^2}{N_\ell^2}N_\ell \sigma_\ell^2(\phi) \\
   =& \sum_{\ell=1}^n \frac{w_\ell^2}{N_\ell} \sigma_\ell^2(\phi) \\
\end{align*}

\subsection{W1:2b}

Lagrange multiplier, with constraint \(\sum_{\ell=1}^n N_\ell = N\):
\begin{align*}
   & L(N_1,\ldots,N_n,\lambda) \\
   =& \text{Var}(\tilde\tau_N)+\lambda\left(\sum_{\ell=1}^n N_\ell - N\right) \\
   =& \sum_{\ell=1}^n \frac{w_\ell^2\sigma_\ell^2(\phi)}{N_\ell}+\lambda\left(\sum_{\ell=1}^n N_\ell - N\right) \\
\end{align*}

\begin{align*}
   & \frac{\partial}{\partial N_\ell} L(N_1,\ldots,N_n,\lambda) = 0 \\
   \Rightarrow& \frac{\partial}{\partial N_\ell} \left(\sum_{\ell'=1}^n \frac{w_{\ell'}^2\sigma_{\ell'}^2(\phi)}{N_{\ell'}}+\lambda\left(\sum_{\ell'=1}^n N_{\ell'} - N\right)\right) = 0 \\
   \Rightarrow&  \frac{\partial}{\partial N_\ell}\frac{w_{\ell}^2\sigma_{\ell}^2(\phi)}{N_{\ell}}+\lambda \frac{\partial}{\partial N_\ell}N_{\ell} = 0 \\
   \Rightarrow&  -\frac{w_{\ell}^2\sigma_{\ell}^2(\phi)}{N_{\ell}^2}+\lambda  = 0 \\
   \Rightarrow& N_\ell^2 = \frac{w_{\ell}^2\sigma_{\ell}^2(\phi)}{\lambda} \\
   \Rightarrow& N_\ell = \frac{w_{\ell}\sigma_{\ell}(\phi)}{\sqrt\lambda} \\
\end{align*}
\begin{align*}
   & \lambda = 0 \\
   \Rightarrow& \sum_{\ell'=1}^n N_{\ell'} - N = 0 \\
   \Rightarrow& \sum_{\ell'=1}^n \frac{w_{\ell'}\sigma_{\ell'}(\phi)}{\sqrt\lambda} - N = 0 \\
   \Rightarrow& \frac{1}{\sqrt\lambda}\sum_{\ell'=1}^n w_{\ell'}\sigma_{\ell'}(\phi) = N \\
   \Rightarrow& \sqrt{\lambda} = \frac{1}{N}\sum_{\ell'=1}^n w_{\ell'}\sigma_{\ell'}(\phi) \\
   \Rightarrow& \lambda = \frac{1}{N^2}\left(\sum_{\ell'=1}^n w_{\ell'}\sigma_{\ell'}(\phi)\right)^2 \\
\end{align*}
\begin{align*}
   & N_\ell = \frac{w_{\ell}\sigma_{\ell}(\phi)}{\sqrt\lambda} \\
   \Rightarrow& N_\ell = \frac{w_{\ell}\sigma_{\ell}(\phi)}{\sqrt{\frac{1}{N^2}\left(\sum_{\ell'=1}^n w_{\ell'}\sigma_{\ell'}(\phi)\right)^2}} \\
   \Rightarrow& N_\ell = N\frac{w_{\ell}\sigma_{\ell}(\phi)}{\sum_{\ell'=1}^n w_{\ell'}\sigma_{\ell'}(\phi)} \\
\end{align*}

\subsection{W1:2c}

\begin{align*}
   & \sigma^2(\phi) \\
   =& \text{Var}(\phi(X)) \\
   =& E[(\phi(X)-E[\phi(X)])^2] \\
   =& E[(\phi(X)-\tau)^2] \\
   =& \int_\text{X} (\phi(x)-\tau)^2 f(x)dx \\
   =& \sum_{\ell=1}^n\int_{\text{X}_\ell} (\phi(x)-\tau)^2 f(x)dx \\
   =& \sum_{\ell=1}^n\int_\text{X} (\phi(x)-\tau)^2 f(x)\mathds{1}_{\text{X}_\ell}(x)dx \\
   =& \sum_{\ell=1}^n w_\ell\int_\text{X} (\phi(x)-\tau)^2 \frac{1}{w_\ell}f(x)\mathds{1}_{\text{X}_\ell}(x)dx \\
   =& \sum_{\ell=1}^n w_\ell\int_\text{X} (\phi(x)-\tau)^2 f_\ell(x)dx \\
   =& \sum_{\ell=1}^n w_\ell\int_\text{X} (\phi(x)-\tau^{(\ell)}+\tau^{(\ell)}-\tau)^2 f_\ell(x)dx \\
   =& \sum_{\ell=1}^n w_\ell\int_\text{X} (A+B)^2 f_\ell(x)dx \\
   =& \sum_{\ell=1}^n w_\ell\int_\text{X} (A^2+2AB+B^2)f_\ell(x)dx \\
   =& \sum_{\ell=1}^n w_\ell\int_\text{X} ((\phi(x)-\tau^{(\ell)})^2+2(\phi(x)-\tau^{(\ell)})(\tau^{(\ell)}-\tau)+(\tau^{(\ell)}-\tau)^2)f_\ell(x)dx \\
   =& \sum_{\ell=1}^n w_\ell\left(\int_\text{X} (\phi(x)-\tau^{(\ell)})^2 f_\ell(x)dx+2\int_\text{X}(\phi(x)-\tau^{(\ell)})(\tau^{(\ell)}-\tau)f_\ell(x)dx+\int_\text{X}(\tau^{(\ell)}-\tau)^2f_\ell(x)dx\right) \\
   =& \sum_{\ell=1}^n w_\ell\left(\sigma_\ell^2(\phi)+2\int_\text{X}(\phi(x)-\tau^{(\ell)})(\tau^{(\ell)}-\tau)f_\ell(x)dx+\int_\text{X}(\tau^{(\ell)}-\tau)^2f_\ell(x)dx\right) \\
   =& \sum_{\ell=1}^n w_\ell\left(\sigma_\ell^2(\phi)+2\int_\text{X}(\phi(x)-\tau^{(\ell)})(\tau^{(\ell)}-\tau)f_\ell(x)dx+(\tau^{(\ell)}-\tau)^2\int_\text{X}f_\ell(x)dx\right) \\
   =& \sum_{\ell=1}^n w_\ell\left(\sigma_\ell^2(\phi)+2\int_\text{X}(\phi(x)-\tau^{(\ell)})(\tau^{(\ell)}-\tau)f_\ell(x)dx+(\tau^{(\ell)}-\tau)^2\right) \\
   =& \sum_{\ell=1}^n w_\ell\left(\sigma_\ell^2(\phi)+2(\tau^{(\ell)}-\tau)\int_\text{X}(\phi(x)-\tau^{(\ell)})f_\ell(x)dx+(\tau^{(\ell)}-\tau)^2\right) \\
   =& \sum_{\ell=1}^n w_\ell\left(\sigma_\ell^2(\phi)+2(\tau^{(\ell)}-\tau)\left(\int_\text{X}\phi(x)f_\ell(x)dx-\tau^{(\ell)}\int_\text{X}f_\ell(x)dx\right)+(\tau^{(\ell)}-\tau)^2\right) \\
   =& \sum_{\ell=1}^n w_\ell\left(\sigma_\ell^2(\phi)+2(\tau^{(\ell)}-\tau)\left(\tau^{(\ell)}-\tau^{(\ell)}\right)+(\tau^{(\ell)}-\tau)^2\right) \\
   =& \sum_{\ell=1}^n w_\ell\left(\sigma_\ell^2(\phi)+(\tau^{(\ell)}-\tau)^2\right) \\
   =& \sum_{\ell=1}^n w_\ell \sigma_\ell^2(\phi)+\sum_{\ell=1}^n w_\ell(\tau^{(\ell)}-\tau)^2 \\
\end{align*}

\subsection{W1:2d (draft)}

Want to show
\[\text{Var}(\tilde{\tau}_N) \leq \text{Var}(\tau_N)\]

\begin{align*}
   & \text{Var}(\tilde\tau_N) \\
   =& \text{Var}\left(\sum_{\ell=1}^n w_\ell \tau_N^{(\ell)}\right) \\
   =& \sum_{\ell=1}^n w_\ell^2 \text{Var}(\tau_N^{(\ell)}) \\
   =& \sum_{\ell=1}^n w_\ell^2 \frac{\sigma^2(\phi)}{N_\ell} \\
   =& \sigma^2(\phi)\sum_{\ell=1}^n w_\ell^2 \frac{1}{N_\ell} \\
\end{align*}


\subsection{W1:3 (draft)}

\begin{align*}
   & F(x) \\
   =& \int_{-\infty}^x \frac{e^{-y}}{(1+e^{-y})}dy \\
   =& \left[\frac{1}{1+e^{-y}}\right]_{-\infty}^x \\
   =& \frac{1}{1+e^{-x}} \\
\end{align*}

\begin{align*}
   & F^{\leftarrow}(x) \\
   =& \ln(\frac{1-u}{u}) \\
\end{align*}

\subsection{W1:4 (draft)}

No standard inverse in this case, so let's use the generalized inverse.
\[F^{\leftarrow}(u) = \inf \{x:F(x) \geq u\}\]

Table-look-up method:

\begin{align*}
   & u & [0,p(0)] & [p(0),p(0)+p(1)] & [p(0)+p(1)), p(0)+p(1)+p(2)] \\
   & F^{\leftarrow}(u) & 0 & 1 & 2 \\
\end{align*}


\subsection{W1:8 (draft)}

Make Taylor expansion of \(\phi(\tau_N)\) around \(\phi(\tau)\).

\[\phi(\tau_N) = \phi(\tau) + \phi'(\zeta)(\tau_N - \tau)\]

\(\zeta\) belongs to the line segments between \(\tau_N\) and \(\tau\).

\[\Rightarrow \sqrt{N}(\phi(\tau_N) - \phi(\tau)) = \phi'(\zeta)\]



























\end{document}
